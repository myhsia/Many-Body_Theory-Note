% !TeX root = ../main.tex

\chapter{Week Coupling (Canonical) Perturbation Theory}

Starting from the equation of motion of the Green's function
\begin{equation}
  \iu\pdif t G_\epsilon [\underset{s_i^+}{c_i}, \underset{s_j^-}{c_j^\dagger}]
= \delta_{ij}
\end{equation}
where the canonical means that the RHS is a $\delta$-function,
and $[s_i^+, s_j^-] = 2s_i^z\delta_{ij}$.
the operators satisfiy Wick's Theorem
\[
  \{c_i, c_j^\dagger\} = \delta_{ij}, \quad [a_i, a_j^\dagger] = \delta_ij
\]

\section{Wick Theorem}

For Gell-Mann-Low, the limitation
\[
  \lim_{\alpha\to0} \frac{u_\alpha^D(0,-\infty)\ket|\eta_0>}
    {\braket<\eta_0|u_\alpha^D(0,-\infty)|\eta_0>}
= \lim_{\alpha\to0} \frac{\ket|\Psi_\alpha^D(0)>}
    {\braket<\eta_0|\Psi_\alpha^D(0)>}
= 
\]
we can convert to the obverse $\braket<A^\text H(t)>$ by computing from the
GML theory
\begin{multline}
  \braket<E_0|A^\text H(t)|E_0>
= \lim_{\alpha\to0} \frac1{\braket<\eta_0|S_\alpha|\eta_0>}
  \sum_{n=0}^\infty \frac1{2^nn!}
  \ab(-\frac\iu\hbar)^n \upe^{-\alpha(|t_1|+\cdots+|t_n|)} \\ \times
  \int_{-\infty}^\infty \d t_1 \cdots \d t_n
  \braket<\eta_0|T_\epsilon
  \ab\{\hat V(t_1)\hat V(t_2) \cdots \hat V(t_n) A^\text H(t)\}|\eta_0>
\end{multline}
where
\[
  V = \frac12 \sum_{\substack{\alpha\beta\\\gamma\delta}}
    \braket<\alpha\beta|\hat V|\gamma\delta>
    a_\alpha^\dagger a_\beta^\dagger a_\gamma a_\delta
\]
The obverse
\[
  \braket<\alpha\beta|\hat V\gamma\delta>
= \braket<\beta\alpha|\hat V|\delta\gamma>
\]
For any observables
\[
  \braket<E_0|A(t) B(t^*) \cdots |E_0>
\]
which can be converted to
\[
  \cdots \int_{-\infty}^\infty \d t_1 \cdots \d t_n
  \braket<\eta_0|
  T_\epsilon\{V(t_1) \cdots V(t_n)
    \underset{a_i(t_i)}A(t) \underset{a_f^\dagger(t_f)}{B(t')} \cdots\}|\eta_0>
\]
To calculate it, consider
\[
  T_\epsilon(O_1(t_1)O_2(t_2)O_3(t_3)O_4(t_4))
\]
which has the order number $4! = 24$.

\section{Normal product}

The ``book-keeping'' of counting:
For a given $\ket|\eta_0>$, the fermi wave vector $k_F$, then
\[
  \gamma_k = \begin{cases}
    c_k^\dagger \ket|\eta_0> = 0, & |k| < k_F\\
    c_k \ket|\eta_0> = 0, & |k| > k_F
  \end{cases}
\]
where $\gamma_k\ket|\eta_0> = 0$.
Also for $\gamma_k^\dagger\ket|\eta_0>$.

For $N$-ordering: If we have an arbitrary sequence
\[
  N(\gamma_1 \gamma_2^\dagger \cdots \gamma_n)
  \to (\gamma^+\gamma^+ \cdots | \gamma \gamma \gamma)
\]
where some of the $\gamma$s have $\dagger$.
The normal ordering operator $N$ brings the $\dagger$ to the front.
\[
  \braket<\eta_0|N(\cdots) |\eta_0> = 0
\]
\begin{example}
  $N(\gamma_1\gamma_2^\dagger \gamma_3) = (-1) \gamma_2\gamma_1\gamma_3
= (-1)^2 \gamma_2^\dagger \gamma_3 \gamma_1$.
\end{example}
\begin{example}
  \[c_1c_2c_2^\dagger c_3^\dagger
= (-1)^3c_3^\dagger c_1c_2c_2^\dagger
= (-1)^5c_3^\dagger c_2 c_2^\dagger c_1
= (-1)^5c_3^\dagger (1 - c_2^\dagger c_2)c_1
= (-1)^6 (c_3^\dagger c_2^\dagger c_2 c_1 + (-1)^5(c_3^\dagger\delta_{22}c_1))\]
So, the normal ordering should be
\[
  \underset{\text{operator manipulation}}{N(\cdots)}
= ()c^\dagger c^\dagger \cdots c + c^\dagger \cdots c + () \cdots
\]
\end{example}

\section{Contraction}

Define the contraction of two operators
\begin{equation}
  \underbrace{A(t) B(t')}
= T_\epsilon(A(t) B(t')) - \mathcal N(A(t)B(t'))
\end{equation}
The operator identity
\[
  \braket<\eta_0|T_\epsilon(A(t)B(t') - N(A(t)B(t')))|\eta_0>
\]
\section{Operator level of ``contraction''}

\begin{example}[Two operators]
  We just look at the RHS of the contraction
  \[
    \underbrace{\gamma(t) \gamma^\dagger(')}
  = T_\epsilon(\gamma(t) \gamma^\dagger(t')) - N(\gamma(t) \gamma^\dagger(t'))
  \]
  On the operator level. Where
  \[
    \theta(t - t') \gamma(t) \gamma^\dagger(t')
  - \theta(t' - t)\gamma^\dagger(t') \gamma(t)
  - [\theta(t - t') + \theta(t' - t)] \gamma^\dagger(t') \gamma(t)
  = \theta(t - t') \gamma(t) \gamma^\dagger(t')
  \]
  Then, we have to define the actual contraction
  \[
    \braket<\eta_0|\underbrace{\gamma \gamma^\dagger}|\eta_0>
  = \iu G^{0,c}[\gamma, \gamma^\dagger]
  \]
\end{example}

Consider 4 time-ordered term
\[
  1\ 2\ 3\ 4:\
  O_1(t_1) O_2(t_2) O_3(t_3) O_3(t_4)
\]
have $4! = 24$ T-ordering: complete permutation of index.
Only look at the two terms
\[
  1234 + 2134 + \cdots
= (12 + 21)34 + T_\epsilon (12)(34)
\xlongequal{\text{Contraction tricks}}
  (\overbrace{12} + N(12)) (34)
\]
So, the observable
\[
  \braket<\eta_0|\quad|\eta_0> = \iu g_{12}^{0c}(34)
+ \braket<\eta_0|N(12)(34)|\eta_0>
\]
So, the two terms gives
\[
  1234 + 2134 \to \iu g_{12}(34), \quad
  1243 + 2143 \to \iu g_{12}(43), \quad
\]
The combines to $\iu g_{12}$ and $\iu g_{34}$,
i.e., $T_\epsilon(\overbrace{12}\overbrace{34})$.
Also for $1324 + 3124$ and $1342 + 3142$, which can be combined into
$T_\epsilon(1234)$. (overbrace 13 and 24).
The LHS have $4!$ terms in total, and we will have $4\times3/2 = 6$ contractions
\[
  g_{12} g_{34} \quad g_{13} g_{24} \quad g_{14} g_{23}
\]
We shall compute
\begin{multline}
  T_\epsilon(123 \cdots 2N)
= N(12\cdots 2N) + \binom{N\text{-product}}{\text{with ONE contraction}}
  N(\underbrace{12}34 \cdots)
+ N(\underbrace{13}24 \cdots 2N) \\
+ \binom{N\text{-product}}{\text{with TWO contraction}}
+ N(\underbrace{12}\underbrace{34}56 \cdots 2N) + \cdots
+ \{\text{total contractions}\}
\end{multline}
\begin{example}[$T_\epsilon(123456)$]
  Apply the result of $1234$
  \[
    T_\epsilon(1234) (56), \quad
    T_\epsilon(C_6^4) \times (C_6^2)
    \to C_6^4 \times \res_{1243} \times (C_6^2)
  \]
  i.e., pick $4$ operators out of $6$.
\end{example}
Do the induction $2n$/$2n + 2$-Wick's
\[
  (\underbrace{12}(34) + N(12)(34)) \cdot
  (\underbrace{56} + N(56)) \to
  \prod_{1, \cdots, 1}^n (\underbrace{i_1 i_2} + N(\\quad))
\]
Together, we have $n$-terms
\[
  1; m \underbrace{} N_{m+1, n}(\quad)
\]
we can put into normal-ordering of arbitrary terms
\[
  \gamma\gamma^\dagger \cdots \gamma^\dagger
= \gamma^\dagger \gamma^\dagger(\delta_{ij} - \gamma^\dagger\gamma)
\]
where we consider $\delta_{ij}$ as one-contruction, and $\ket|\eta_0>$ is the
ground state.

Underlying structure $N(\quad) \ket|\eta_0> = 0$ for Wick's theorem
($\ket|\eta_0>$ Gaussion states).

The canonical algebra
\[
  \{c_i, c_j^\dagger\} = \delta_{ij}, \quad
  \underbrace{\gamma \gamma^\dagger} = T_\epsilon - N(\cdots)
  \to s^2 \Tr(s^+, s^-)
\]
One is canonical algebra, and one is the Gaussion states.
The main idea is to convert $\{s^+, s^-\}$ into canonical bosons:
Schwinger Boson, Maleeev, Pvimakoff.

For $A = \identity$, it becomes
\[
  \fbox{\dots} \Rightarrow \braket<\eta_0|U_\alpha(t,t')|\eta_0>_{t,t'\to\infty}
\]
and $S_\alpha = U_\alpha(+\infty, -\infty)$.
Do the Wick's rotation $t\to\iu\tau$, $S_\alpha \to Z$.
$\braket<\Psi_0|\Psi_0>$ and $\braket<\eta_0|\Psi_D>$.

\section{Feynman Diagram (of what)}

\section{$\braket<\eta_0|U|\eta_0>$: Vacuum diagrams}

We rewrite the term into Taylor series form
\begin{equation}
  \braket<\eta_0|U|\eta_0>
= 1 + \sum_{n=1}^\infty \braket<\eta_0|U_\alpha^{(n)}(t,t')|\eta_0>
\end{equation}
For $n = 1$
\[
  V(t_1) = \frac12 \sum_{\substack{kl\\nm}} \nu(kl, nm)
  \int \d t_1' \delta(t_1 -t_1')
    a_k^\dagger(t_1) a_l^\dagger(t_1') a_m(t_1') a_n(t_1)
\]
where the indes is just the matrix element $\braket<kl|v|nm>$.
\begin{align*}
  \braket<u^{(1)}> {=} & \frac12 \sum_{\substack{lk\\mn}} \nu(\cdots)
  \int \d t'\delta(t_1 - t_1')
  \braket<\eta_0|
    T_\epsilon(a_k^\dagger(t_1)a_l^\dagger(t_1')a_m(t_1')a_n(t_1))|\eta_0>\\
{=} & (-\iu G^{0,C}(a_k^\dagger, a_n)[t_1 - t_1])
      (-\iu G^{0,c}(a_0^\dagger, a_m) [t_1' - t_1'])\\
+ & (-1)(-\iu G^{0,c}(a_k^\dagger, a_m)[t_1 - t_i])
    (-\iu G^{0,c} [a_l^\dagger, a_n](t_1' - t_1))
\end{align*}
\begin{center}
  \includegraphics[width = \linewidth]{./media/feynmann1.jpeg}
\end{center}
The operators and the vertices
$
  a_k^\dagger:
  \tikz[baseline]
    { \fill (0,0) circle (.1); \draw [->] (0,0) -- (1,0);} \quad
  a_k:
  \tikz[baseline]
    { \fill (0,0) circle (.1); \draw [<-] (0,0) -- (1,0);}
$
which form through the labeled Feynman diagram.

\section{Feynman rules for labeled Feynman diagrams}

\begin{equation}
  V = \frac12 \int \d t_1' \delta(t - t_1') V_{klmn}
      a_k^\dagger(t_1) a_l^\dagger(t_1') a_n(t_1')a_n(t_1)
\end{equation}
\begin{center}
  \includegraphics[width = \linewidth]{./media/feynmann2.jpeg}
\end{center}
For $\braket<lk|\hat V|mn>$, if one switch $\braket<k(L)l(R)|\hat V|n(L)m(R)>$,
then the diagram will be flipped: permuting extremities($L \leftrightarrow R$).
\begin{enumext}[columns = 2]
  \item Vertex i: $\nu(k_i l_i; m_in_i)$
  \item Propagating line:
  \[
    \tikz[baseline]{
      \fill (0,0) coordinate (a) circle (.1) node [below] {$t_i$};
      \fill (1,0) coordinate (b) circle (.1) node [below] {$t_j$};
      \draw (a) to [bend left = 20] (b);
    } \quad
    -\iu G_{k_i}^{0,c} (t_i - t_j) \delta_{k_ik_j}
  \]
  \item Non-propagating line:
  \[
    \tikz[baseline]{
      \node [left] at (-.25,0) {$t_i$};
      \fill (-.25,0) circle (.05);
      \draw [->] (-.25,0) arc (-180:0:.25);
      \draw (0,0) circle (.25);
  } \quad
    G^{0,c}(t_j - t_i)
  \]
  \item $(-1)^S$, $s$\# fermion loop.
  \item Summ order dummy \dots
  \item $\exp(-\alpha(|t_1| + |t_2| + \cdots))$
  \item Integrate $t$ \dots
  \item $\frac1{2^nn!} \ab(-\frac\iu\hbar)^n$
\end{enumext}

\section{Unlabeled Feynman diagram \# of labeled diagram $\mathrm{(2\eta)!}$}

\begin{center}
  \includegraphics[width = \linewidth]{./media/feynmann3.jpeg}
\end{center}
we can integrate over
\begin{equation}
  G(t_1 - t_3) G(t_1 - t_2) G(t_2 - t_3)
\end{equation}
in these Feynman diagrams.

\section{Remove labels, get right results}

\begin{enumext}
  \item Diagram with same ``topology'' (means classification in math)
  hvae the same value
  \item Given all top. inequ diagram how we count them?
\end{enumext}
\begin{equation}
  \sum_{n_0}\{\text{All contractioin} =\}
\end{equation}
\begin{center}
  \includegraphics[width = \linewidth]{./media/feynmann4.jpeg}
\end{center}

Transformations of $\Gamma \to \Gamma'$,
but leave  it's value unchanged.
\begin{center}
  Symmetry factor.
\end{center}
We put the labes backing: $k$, $l$, $n$, $m$.

\paragraph{Transformation}
Leaves two structures
\begin{enumext}[columns = 2]
  \item Value unchanged
  \item Diff contraction.
\end{enumext}
Two type of transformations ([Ref] Coloman: symmetry factor):
\begin{enumext}[columns = 2]
  \item Permute extremities.
  \item Propagating lines.
\end{enumext}
\begin{center}
  \includegraphics[width = \linewidth]{./media/feynmann5.jpeg}
\end{center}
If we have $n$ vertices and $m$ lines, and this will give
$2^nn!m!$ possibilities.

All these transf leave value unchange, but
\begin{enumext}[columns = 2]
  \item Different contraction (labeled diag)
  \item Same contraction \texttt{->} over counting
\end{enumext}
The sum
\[
  \sum\{\text{All contractions}\}
\]
Symmetry factor: account for overcounting.
The flips come from the transformations
\begin{center}
  \includegraphics[width = \linewidth]{./media/feynmann6.jpeg}
\end{center}

\paragraph{\# Final step}
Obtain the symmetry factor $S$.
\begin{enumext}
  \item All pass: value-inv.transf $G$.
  \item $G_\Gamma$: Transf $\Gamma$ into a deformations (same contraction)
  of itself. Hence, $G_\Gamma$ is sub group of $G$.
  \item $S = \#$ of elements of $G_r$, $S$ must be divisor of $2^nn!m!$.
\end{enumext}

\section{Unlabeled Feyman diagram}

Using label permutation to obtain $S$, $\Gamma'$
\[
  G_\Gamma: \mathcal G\{(12), (21)\}
\]
means $1 \to 2$ and $2 \to 1$.
\begin{center}
  \includegraphics[width = \linewidth]{./media/feynmann7.jpeg}
\end{center}
These diagrams have $S = 2$.
\begin{example}
  Two vertices diagram $\mathcal G\{(1234), (3412)\}$. Using label Pertubation:
  $1 \to 3$, $2 \to 4$, $3 \to 1$, $4 \to 2$.
\end{example}
\begin{example}[$(43\ 12) \notin \mathcal G$]
  Diff contraction
\end{example}
\begin{center}
  \includegraphics[width = \linewidth]{./media/feynmann8.jpeg}
\end{center}
\begin{example}
  Compare with two diagrams
  \[
    \mathcal G\{(1234), (3412), (2143), (4321)\} \to S = 4
  \]
  COmbines them Topo ineq.
  \[
    G_\Gamma \to S, \quad
    \cancel{\frac1{2^nn!m!}} \sum\{n||\text{contractionss}\} = \frac1s (\cdots)
  \]
  which cancels by ``All poss value-inv-transf'', for $v = \frac12$.
\end{example}